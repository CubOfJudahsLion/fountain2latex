\documentclass[11pt]{article}
\usepackage{hyperref}
\usepackage[dvipsnames]{xcolor}
\usepackage[margin=1in]{geometry}
\newcommand{\link}[2]{\textcolor{Blue}{\href{#1}{#2}}}
\newcommand{\cmdline}[1]{    \hspace{0.5in}\colorbox{Black}{\textcolor{White}{\texttt{#1}}}}
\definecolor{Parchment}{RGB}{231,211,178}
\setlength{\parskip}{6pt plus2pt}
\setlength{\parindent}{0pt}
\title{\textbf{\texttt{fountain2latex}}}
\author{Alexander Feterman Naranjo}
\date{July 2023}

\begin{document}
\maketitle

\texttt{fountain2latex} is a simple utility to convert from the
\texttt{\link{https://fountain.io/}{.fountain}} screenplay format to
the \link{https://www.latex-project.org/}{\LaTeX} screenplay subformat.
{\LaTeX} provides superior typesetting for professional-looking
documents.

This utility relies on the
\link{https://www.ctan.org/pkg/screenplay}{screenplay {\LaTeX} class}
and the \link{https://www.ctan.org/pkg/screenplay-pkg}{\texttt{screenplay-pkg} package},
both by \link{http://dvc.org.uk/johnny.html}{John Pate}. The former
implements a \texttt{document class} with Academy-recommended rules and
macros; the latter creates an \texttt{enviroment} which allows inclusion
of screenplay-formatted sections in larger documents. They're both
available in the \link{https://tug.org/texlive/}{TeXLive} and
\link{https://miktex.org/}{MiKTeX} distributions, both of which offer
packages for Linux, Windows and Mac.

\texttt{fountain2latex} is written in \link{https://haskell.org}{\textbf{Haskell}}
and compiled with \textsf{GHC} version 9.2.8.


\section*{Using \texttt{fountain2latex}}

When it's installed (and in your path), \texttt{fountain2latex} can be
invoked in several ways. \emph{Legend}: braces (\texttt{\{\}}) mean you
must select one of the alternatives separated by vertical bars (\texttt{|}).
Brackets (\texttt{[]}) mean the alternatives are optional.
\begin{itemize}
  \item{To display the version:\\ \\
    \cmdline{fountain2latex \{-v|-{}-version\}}}\\
  \item{To see basic usage instructions:\\ \\
    \cmdline{fountain2latex \{-u|-{}-usage\}}}\\
  \item{To get a (somewhat) more comprehensive help text:\\ \\
    \cmdline{fountain2latex \{-h|-{}-help\}}}\\
  \item{To convert standard input into standard output:\\ \\
    \cmdline{fountain2latex [-p|-{}-as-part]}\\ \\
    The argument \texttt{-p} (or its equivalent longer variant, \texttt{-{}-as-part})
    instructs \texttt{fountain2latex} not to generate a standalone document,
    but to emit a document fragment that can be included in another. The master
    document must use \texttt{screenplay-pkg}:\\ \\
    \colorbox{Parchment}{\textcolor{Black}{\texttt{\textbackslash{}usepackage\{screenplay-pkg\}}}}}\\
  \item{To convert a file to standard output:\\ \\
    \cmdline{fountain2latex [-p|-{}-as-part] \emph{<input-file>}[.fountain]}\\ \\
    \texttt{-p} (or \texttt{-{}-as-part}) works exactly the same as in the previous
    case. If the \texttt{.fountain} extension is omitted, \texttt{fountain2latex}
    will try to find a file with the same name as the argument; failing that, it
    will append \texttt{.fountain} to it and then try again.}
  \item{To convert a file into another:\\ \\
    \cmdline{\small{fountain2latex [-p|-{}-as-part] \emph{<input-file>}[.fountain] \{.|\emph{<output-file>}[.tex]\}}}\\ \\
    \texttt{-p} and the optional \texttt{.fountain} extension work as
    stated above. For the output file, if the \texttt{.tex} extension
    is omitted, it will be added. If you use a period instead of an
    output filename, the output file will be named the same as the
    input one, with the extension changed to \texttt{.tex}.}
\end{itemize}

More accurately, \texttt{fountain2latex} processes all options from
left to right, each option overriding previous ones.


\section*{Why?}

I find {\LaTeX} to be more obtrusive to my writing flow, but its
typesetting is without par. Fountain barely requires an extra
character here or there, so it's more amenable to creative flow
and it's plain text so I can use even my favorite
\link{https://www.vim.org/}{code editor} (with
a \link{https://github.com/kblin/vim-fountain}{helper plugin}
for some extra niceties) but no formatting magic.

That's where \texttt{fountain2latex} comes in: a simple console
application. Just a few keystrokes in your terminal and that's it.
Zero leak risks.


\section*{Installing from release}

You know this one. Download the latest release, extract all the files
to any directory, optionally add the directory to your \texttt{\$PATH}
for convenience.


\section*{Installing from sources}

Needless to say, you need \textsf{GHC} 9.2.8+ to do this. If you don't
have it, you can always install from release, as shown in the previous
section.

Just \texttt{cd} to the \texttt{fountain2latex} directory and run:\\
\\
\cmdline{make install}\\
\\
Which should take care of everything, including overwriting older
versions.


\section*{Contact}

I can be reached at \texttt{\link{mailto:10951848+CubOfJudahsLion@users.noreply.github.com}{10951848+CubOfJudahsLion@users.noreply.github.com}}.\\

\end{document}


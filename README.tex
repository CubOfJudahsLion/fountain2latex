\documentclass[11pt]{article}
\usepackage{hyperref}
\usepackage[dvipsnames]{xcolor}
\newcommand{\link}[2]{\textcolor{Blue}{\href{#1}{#2}}}
\author{Alexander Feterman Naranjo}
\date{July 2023}
\title{fountain2latex}

\begin{document}
\maketitle

\begin{center}
\scriptsize{\emph{(Click the back arrow to return to the repo)}}
\end{center}

\texttt{fountain2latex} is a simple utility to convert from the
\texttt{\link{https://fountain.io/}{.fountain}} screenplay format to
the \link{https://www.latex-project.org/}{\LaTeX} screenplay subformat.
{\LaTeX} provides world-class typesetting quality for that extra bit of
class and professionalism in your script.

To be more precise, this utility specifically relies on the
\emph{\link{https://www.ctan.org/pkg/screenplay}{screenplay {\LaTeX} class}}
by \textbf{John Pate}, which implements Academy-recommended rules. It can be
found in the \link{https://tug.org/texlive/}{TeXLive} and
\link{https://miktex.org/}{MiKTeX} distributions, both of which offer
packages for Linux, Windows and Mac.

\texttt{fountain2latex} is written in \link{https://haskell.org}{Haskell}
and compiled with GHC version 9.2.8.


\section*{Why?}

I find \LaTeX to be more obtrusive to my writing flow. Fountain barely
requires an extra character here or there. Having to think about the
\texttt{command} I need to properly format something is disrupting.
Then again, I want the sweet, sweet typesetting...

There's gotta be a way to have my cake and eat it too.

\emph{``But Cub, you beautiful, daunting force of nature''}, I hear you say,
\emph{``there are already ways to convert fountain to other formats.''}

I know, but I'm old school (I'm actually using \emph{make} here.)
Something this simple should be fast --- a few keystrokes in your
terminal and that's that. It certainly shouldn't require a danged
cloud API.

Now where did I leave that gosh-darned Bengay.


\section*{Installing from release}

Download the latest release and extract all the files to any directory.
Run by typing\\
\\
\texttt{    fountain2latex \emph{<input>}[.fountain] [\emph{<output>}[.tex]]}\\
\\
Extensions can be omitted, and the result will be sent to standard
output if the second filename is not provided.


\section*{Installing from sources}

Needless to say, you need GHC 9.2.8+ to do this. If you don't have it,
you can always install from release, as shown in the previous section.

Just \texttt{cd} to the \texttt{fountain2latex} directory and run:\\
\\
\texttt{    make install}\\
\\
Which should take care of everything, including overwriting older
versions.


\section*{Contact}

I can be reached at \link{mailto:10951848+CubOfJudahsLion@users.noreply.github.com}{10951848+CubOfJudahsLion@users.noreply.github.com}.

\end{document}

